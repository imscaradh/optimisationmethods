\documentclass[10pt]{article}         %% What type of document you're writing.

%%%%% Preamble

%% Packages to use
\usepackage{amsmath,amsfonts,amssymb}   %% AMS mathematics macros

%% Title Information.
\title{Bericht Optimierungsmethoden}
\author{Timon Tschanz, Pascal Zingg}
%% \date{}           %% By default, LaTeX uses the current date

%%%%% The Document

\begin{document}

\maketitle

\section{Einführung}
Im Rahmen des Kurses Analysis haben wir uns mit verschiednen Optimierungsaufgaben vertraut gemacht. Die grundlegende Technik, die wir dazu verwendeten, war die 1. Ableitung gleich 0 zu setzen, damit wir ein optimales Ergebnis für ein konkretes Problem erhielten. In diesem Bericht befassen wir uns näher mit Algorithmen, welche es ermöglichen Extremalstellen zu berechnen. Zudem diskutieren wir anhand von diversen Aufgabenstellungen, wie diese zur Anwendung kommen können. Durch diverse Visualisierungen versuchen wir mit diesem Bericht versuchen wir die Ergebnisse bildlich darzustellen, was aus unserer Sicht auch zu einer glaubwürdigeren Diskussion führen wird.

\section{Motivation}
Da es in unserer Umgebung sehr viele Problemstellungen gibt, bei denen jeweils ein Optimum gesucht ist, sehen wir in dieser Arbeit eine konkrete und relativ einfache Anwendung der Ableitungen. Die Algorithmen, welche wir mit Python umgsetzt haben, sind wiederverwendbar, da wir sie so weit als möglich versuchten, zu parametrisieren. 

\pagebreak
\section{Auswertungsverfahren}
\subsection{Algorithmen}
Um die Extremalstellen einer spezifischen Funktion zu errechnen, gibt es diverse Vorgehensweisen. In diesem Kapitel gehen wir auf die 3 im Unterricht behandelten Verfahren ein und erläutern ihre Funktionsweise. 

\subsubsection{Gradientenverfahren}
Durch diese Methode wird im generellen versucht, sich an eine Extremalstelle anzunähren. $x_0$ wird als Startpunkt definiert. Vorausgesetzt wird hierbei, dass sich der Punkt links von einer Extremalstelle befindet. Veranschaulichen wir uns die nachfolgende quadratische Funktion:
$$
f(x)=x^2
$$


\subsubsection{Nelder-Mead}

\subsubsection{Nullstelle der Ableitung}


\section{Diskussion}

\end{document}
